%%%%%%%%%%%%%%%%%%%%%%%%%%%%%%%%%%%%%%%%%%%%%%%%%%%%%%%%%%%%%%%%%%%%%%%%%%%%%%%%%%%%%%%%%%%%%%%%%%%%%%%%%%%%%%%%%%%%%%%%%%%%%%%%%%%%%%%

\chapter{Introduction}
\label{cha:introduction}

%%%%%%%%%%%%%%%%%%%%%%%%%%%%%%%%%%%%%%%%%%%%%%%%%%%%%%%%%%%%%%%%%%%%%%%%%%%%%%%%%%%%%%%%%%
\section{Background}

Sound surrounds us in all our lives. As a society, we live in a world engulfed by many types of sounds, generated by various aspects of our lives like transportation, commerce and entertainment. These sounds form an environment within our society as well as affect and interact with various segments of our lives.

While some parts of this sound environment in our urban landscape are critical for our daily lives, most parts of this environment are a serious threat to our health, both physical and psychological, and furthermore, impair the possibilities of recreation. In particular, traffic noise pollution is a great and increasing environmental problem. 

Traffic noise has been related to sleep disturbance, and also certain cardiovascular conditions \cite{ref:Bart1}. Due to the adverse health effect of traffic noise, the World Health Organization (WHO) has recognized environmental noise, including traffic noise, as a serious threat to public health. A conservative estimate of the health costs of environmental noise are in the range of � 40 Billion annually in Europe. WHO claims that more than 40\% of the European population is exposed sound levels exceeding the maximum levels published by WHO, which indicates a serious annoyance and health hazard. 

Over the past decades, significant amount of research has been done on ways to reduce the environmental noise and its effect, especially within the urban environment. It has been shown that soundscapes of an urban environment can be made more pleasant and healthy if the planners and designers of these environments can be made aware and are empowered. There needs to be a shift in focus from a quantitative perspective of noise, considering only the level of the noise, towards a qualitative perspective where the observer is considered rather than just an acoustical quantity. However, the qualitative aspects of a sound environment are difficult to communicate, particularly to people without special training. Many times groups of people who can have a major influence on the sound environment through their decisions, e.g. politicians, governmental administrators, experts from building industry and citizens, do not have the required training to be able to measure and communicate the quality of sound. The traditional techniques used to document and communicate sound environments are noise maps, which are rather abstract and difficult to understand for the common population.

To aid in the means for communication of the qualitative aspects of a sound environment, there is a need for a tool, some methods and terminologies which allow effective transfer and understanding of it. There is an urgent need for a better and more understandable representation of the sound environment. 

The LISTEN Project (see e.g. [2]) aims to study these issues and create a software system which is able to demonstrate and allow the stake holders to listen and understand the qualitative aspects of specific soundscapes and the corresponding health effects in particular urban environments.

%%%%%%%%%%%%%%%%%%%%%%%%%%%%%%%%%%%%%%%%%%%%%
\subsection{The LISTEN Project}

The LISTEN project is a three year research collaboration between the Interactive Institute, Kungliga Tekniska H\"{o}gskolan (KTH) -Marcus Wallenberg Laboratory, University College of Arts, Crafts and Design (Konstfack), Stockholm University - G\"{o}sta Ekman Laboratory and Chalmers University of Technology - Applied Acoustics. 

The goal of the project is to build a demonstrator software system for simulation and auralization of the sound environment of an urban area. The purpose of the demonstrator is to show that it is possible to listen to an urban soundscape which is still at the planning stage of development, and also allow one to hear how noise prevention measures, such as a noise barrier, will affect the soundscape.

The project focuses on the simulation of road and rail traffic within urban environments. Various scenarios are considered in this project including various types of road and rail noise, existence and non-existence of noise barriers and positioning of the receiver in backyards as well as inside apartment rooms.

This thesis is a part of the LISTEN project looking at a part of the auralization of sound sources for the urban soundscape.

%%%%%%%%%%%%%%%%%%%%%%%%%%%%%%%%%%%%%%%%%%%%%%%%%%%%%%%%%%%%%%%%%%%%%%%%%%%%%%%%%%%%%%%%%%
\section{Auralization and Audio Synthesis}

The term �Auralization� was first coined by Kleiner \cite{ref:Kleiner1993} to mean �the process of rendering audible, by physical or mathematical modeling, the sound field of a source in a space, in such a way as to simulate the binaural listening experience at a given position in the modeled space�. The ability to recreate listening environments has always been one of the aims of acoustics and audio engineering. The goal has been not only to recreate the sensation of the speech or music, but also allow the recreation of the aural impression of the acoustic characteristics of a space, be it outdoors or indoors.

Auralization (also known as Sonification or Virtual Acoustics) is a very important part of the acoustic designers toolbox and part of a solution for a better sound environment. The ability to recreate on demand any arbitrary soundscapes, allows designers to understand the limitations and practicalities of noise abatement and control. Furthermore, the ability of Auralization techniques to allow the qualitative listening of the soundscape makes it a useful tool.

In the case of the LISTEN project, Auralization is utilized to model the sound sources and simulate the propagation of the sound in the urban landscape for each of the scenarios. Hence, the source modeling and propagation can be considered two of the major areas where Auralization techniques are employed in the LISTEN project.

Modeling of outdoor sound propagation is a matured field of study, with multiple analytical and numerical methods for obtaining the relationship between sound at the source and the receiver. Various environmental effects as well as conditions need to be taken into account to be able to generate these relationships. The propagation studies for LISTEN project were conducted and modeled by Forss\'{e}n et al \cite{ref:listen}.

The sound source modeling has been an area of interest for many fields of Acoustics. From computer musicians to researchers looking at road and tire noise, the modeling of the source of a sound allows one to have an insight into the inherent properties and mechanism of the source of the sound. Once such models are generated, they can be used to synthesize sounds with various source properties, allowing one to simulate sounds generated in response to various changes to the source. This is known as Audio Synthesis.

The term Audio Synthesis (also referred to as Sound Synthesis) was originally used to describe the digital creation of musical sounds by electronic synthesizers. However, with the advent of modern technology and the adoption of such technologies by the musical community, the term Audio Synthesis has expanded from just looking at musical sounds, to synthesis of sounds of all types. 

With the major contribution from the computer music research community, there exist a multitude of techniques of Audio Synthesis, which have been developed over the years each targeted to a specific concept or perspective of looking at sound. Spectral Modeling synthesis \cite{ref:sms}, Frequency Modulation Synthesis \cite{ref:FMS}, Granular Synthesis \cite{ref:granular}, are all different approaches to Audio Synthesis based on various methods of generating and manipulating sounds. Different techniques have been used to model different types of sounds based on the strengths of the technique.

Spectral Modeling Synthesis (SMS) , is a technique which was developed in the early 1990s by Serra, X. at the Center for Computer Research in Music and Acoustics (CCRMA), at Stanford University. This technique was developed for modeling musical instruments and relies on the very tonal sounds made by such instruments. The technique has since been developed further \cite{ref:smsAdv} to allow a wider array of sounds to be modeled, by integrating the abilities to capture various other aspect like noise levels, transients, etc. 

This thesis looks at using Spectral Modeling Synthesis to model vehicles as sound sources to be used in Auralization along with outdoor sound propagation techniques to allow simulation of urban soundscapes.

%%%%%%%%%%%%%%%%%%%%%%%%%%%%%%%%%%%%%%%%%%%%%%%%%%%%%%%%%%%%%%%%%%%%%%%%%%%%%%%%%%%%%%%%%%
\section{Purpose and problem definition}

%%%%%%%%%%%%%%%%%%%%%%%%%%%%%%%%%%%%%%%%%%%%%
\subsection{LISTEN Demonstrator}

The demonstrator created for the LISTEN project is to allow the qualitative listening of urban landscape. The demonstrator has to be able to generate perceptually believable soundscapes, as well as allow the real-time interaction of the user to give a sense of realism \cite{ref:listen}. These two requirements enforce the direction of the development of the demonstrator. 

The demonstrator is implemented in the Pure Data (PD) \cite{ref:listen} programming environment. The calculations for the propagation model are based on the engineering methods from state-of-art noise mapping prediction, the European �Harmonoise� \cite{ref:harmonoise} and the Scandinavian �Nord2000� \cite{ref:nord2000} methods. 

The original design of the demonstrator used pre-synthesized vehicle sounds as source for the noise. These were then put through the propagation model to yield the sound at the receiver location. A large number of vehicle sounds were required to have a database of sounds which could be used to generate the various combination of vehicles (light, medium, heavy), and their travel speeds. While this approach allow both the real-time and the perceptual requirement to be met, the resulting database was significantly large (in the case of a preliminary test on just light vehicles, the database size was 5GB). And to meet the real-time requirements, this database needed to be loaded up into the system RAM, which became a limiting factor to the system's real-time performance. Hence, the database approach did not scale when looking at a more complicated source model comprising of various types of vehicles. Furthermore, while looking at alternative approaches to the deployment of the system in the future, including a possible web-based implementation, such a huge RAM requirement seemed limiting.

Thus, a synthesis based implementation was sought for such a system, which would allow a faster and a more flexible solution that requires less memory. A hybrid Spectral-Granular synthesis model was also experimented with prior to this thesis project. However, due to limited success in modeling the sounds of a bus, a more traditional approach of using just the Spectral Modeling technique was chosen for this thesis.

%%%%%%%%%%%%%%%%%%%%%%%%%%%%%%%%%%%%%%%%%%%%%
\subsection{Audio playback vs. synthesis}

Audio playback and audio synthesis are the two approaches of modeling the source in Auralization. Audio playback just refers to the technique of recording the source in appropriate conditions (in the case of cars, on a specially designed recording track), and use those recordings as source material for the Auralization. The sound propagation model can then be applied to the source recordings based on the source and receiver positions and surroundings, and the final sound at the receiver can be calculated.

The playback approach is simple and elegant, however it does not scale easily if a variety of sources is required. In the case of the LISTEN demonstrator, it was important to incorporate various types of vehicles into the source so a realistic noise source could be used. Also, when trying to generate a source for simulating vehicles traveling at a combination of speeds, a playback approach gets a little unwieldy with the large amount of recorded data required to be able to play back such a combination source.

A synthesis approach has it�s own advantages in Auralization. Synthesis models tend to use much less memory than recordings, hence there�s an inherent data compression, which allows the model based approach to be a lot more flexible. Also, synthesis algorithms can designed to be able to produce sounds in real-time allowing live interaction of the listener with the demonstrator and yet not use up much computational effort.

Furthermore, a synthesis approach allows the manipulation of the sounds produced, without requiring multiple pre-calculated sounds. For example, in the case of a vehicle noise model, if the speed of the vehicle can be extracted as a parameter of the model, then a single model could be used to generate sounds for that type of vehicle traveling at various speeds. Other manipulations are also possible, depending on the synthesis model and technique used and the ability to extract various parameters from them.

%%%%%%%%%%%%%%%%%%%%%%%%%%%%%%%%%%%%%%%%%%%%%%%%%%%%%%%%%%%%%%%%%%%%%%%%%%%%%%%%%%%%%%%%%%
\section{Problem definition}

The main purpose of this thesis work is to develop and tune a Spectral Modeling Synthesis based system for auralizing sounds of a vehicle passage for use in the LISTEN Demonstrator. Using Spectral Modeling Synthesis, implementing an Analysis and Synthesis technique which can be integrated along with the propagation models in to the LISTEN Demonstrator for a realistic and real-time simulation of the urban soundscape.

%%%%%%%%%%%%%%%%%%%%%%%%%%%%%%%%%%%%%%%%%%%%%%%%%%%%%%%%%%%%%%%%%%%%%%%%%%%%%%%%%%%%%%%%
\section{Limitations}

The system developed as a part of this Thesis work has a few limitations relating to the assumptions made about environmental conditions, and also the sources being modeled.

Since motivation for this research work was dealing with vehicle sounds, it was assumed the content of the sound being modeled being very much related to sounds produced by vehicles, with a strong tonal component (engine contribution) and a wide-band noise (tire contribution). Other types of noises, especially transient noises, were ignored from consideration in the system.

%%%%%%%%%%%%%%%%%%%%%%%%%%%%%%%%%%%%%%%%%%%%%%%%%%%%%%%%%%%%%%%%%%%%%%%%%%%%%%%%%%%%%%%%
\section{Thesis structure}

The thesis is structured as follows:

\emph{Chapter Two}: Contains the Theory the Spectral Modeling synthesis technique and how it can be used to Analyze sounds to create sound models and then Synthesize these models back in to sounds.

\emph{Chapter Three}: Presents the implementation details of the system. This chapter looks at how the various parts of the SMS technique and the additions to it were implemented in the system, as well as the details of the sounds model.

\emph{Chapter Four}: Covers at the analysis of the implemented system and also the results of tests done to verify the perceptive realism of the synthesized sounds.

\emph{Chapter Five}: Discusses the implications of the results and the use of such sounds models in the demonstrator system.

\emph{Chapter Five}: Finally, all the results and findings are summarized. In addition, research topics for continuation of this work are discussed.